\documentclass{beamer}\mode<presentation> {\usetheme{Madrid}}\usepackage{booktabs,adjustbox}\usepackage{xcolor,colortbl}\title[RM, EDF, LLF]{Proyecto 2: Calendarizacion en Tiempo Real}\author{Vargas A, Camacho A, Morales V}\institute[TEC]{Tecnologico de Costa Rica \medskip\textit{avargas@gmail.com, acamacho@gmail.com, verny.morales@gmail.com}\textit{3er Cuatrimestre}}\date{\today}\begin{document}\begin{frame}\titlepage\end{frame}\begin{frame}\frametitle{Rate Monotonic}\begin{block}{Tipo}Algoritmo de Scheduling Dinamico, utilizado para la reslucion de problemas caoticos, como por ejemplo el problema de los carros autonomos.\end{block}\begin{block}{Manejo de prioridades}Algoritmo de prioridades estaticas, esto quiere decir que ninguna tarea puede cambiar su prioridad. Donde la prioridad de una tarea siempre es igual a su periodo. Periodo mas corto, mayor la prioridad.\end{block}\begin{block}{Supuestos}Todas las tareas criticas son periodicas, e independientes. El tiempo de computacion se conoce a priori, y el cambio de contexto es igual a cero, o ya esta considerado en el tiempo de computacion.\end{block}\end{frame}\begin{frame}\frametitle{Teoremas de Scheduling}\begin{theorem}[Parametros a tomar en cuenta]$\mu = \Sigma \frac{Ci}{Pi}$ Utilizacion del CPU$ Un = n 2^\frac{1}{n} - 1 $ donde n es la cantidad de tareas\end{theorem}\begin{theorem}[Condiciones de suficiencia]$\mu \leq Un $ Tareas calendarizables $\mu \geq Un$ Debido a que es una condicion de sufuciencia podria ser calendarizable$\mu \succ 1$ Tareas no calendarizables\end{theorem}\end{frame}\begin{frame}\frametitle{Early Deadline First}\begin{block}{Tipo}Algoritmo de Scheduling Dinamico, en donde las tareas son periodicas. Se considera un algoritmo optimo para algoritmos de prioridades dinamicas. Es un algoritmo de tipo expropiativo.\end{block}\begin{block}{Manejo de prioridades}El nombre del algoritmo indica la politica de prioridad. La prioridad es inversamente proporcional al tiempo que falta para el deadline. El deadline de cada tarea es igual al periodo de la misma. \end{block}\begin{block}{Supuestos}Todas las tareas criticas son periodicas, e independientes. El tiempo de computacion se conoce a priori, y el cambio de contexto es igual a cero, o ya esta considerado en el tiempo de computacion.\end{block}\end{frame}\begin{frame}\frametitle{Teoremas de Scheduling}\begin{theorem}[Parametros a tomar en cuenta]$\mu = \Sigma \frac{Ci}{Pi}$ Utilizacion del CPU\end{theorem}\begin{theorem}[Condiciones de suficiencia]$\mu \leq 1 $ Tareas calendarizables, ya que es una condicion necesario y de suficiencia\end{theorem}\end{frame}\begin{frame}\frametitle{Least Laxity First}\begin{block}{Tipo y manejo dr prioridades}Es un algoritmo donde las prioridades se manejan de forma dinamica. Por cada unidad de tiempo se deben de evaluar las prioridades para asi conocer el Laxity de cada tarea, y tomar una decision sobre el scheduling de las mismas.\end{block}\end{frame}\begin{frame}\frametitle{Teoremas de Scheduling}\begin{theorem}[Parametros a tomar en cuenta]$ Laxity = Di - Ti + Ci $ Donde D es el deadline proximo de la tarea, T es el tiempo actual de ejecucion y C es el tiempo computacional faltante de la tarea \end{theorem}\begin{theorem}[Criterio de Scheduling]Se toma el L menor entre todas las tareas en la cola de Ready y se ejecuta la tarea para cada tiempo de ejecucion\end{theorem}\end{frame}
\begin{frame}\frametitle{All Scheduling Results}\begin{table}\adjustbox{max height=\dimexpr\textheight-5.5cm\relax,max width=\textwidth}{\begin{tabular}{l|c|c|c|c|c|c|c|c|c|c|c|c|c|c|c|c|c|c|c|c|c|c|c|c|c|c}
 &0&1&2&3&4&5&6&7&8&9&10&11&12&13&14&15&16&17&18&19&20&21&22&23&24\\ 
 \hline 
Task 0& &\cellcolor{blue}&\cellcolor{blue}& & & & &\cellcolor{blue}&\cellcolor{blue}& & & & &\cellcolor{blue}&\cellcolor{blue}& & & & &\cellcolor{blue}&\cellcolor{blue}& & & & &\\ 
Task 1& & & &\cellcolor{green}&\cellcolor{green}& & & & &\cellcolor{green}&\cellcolor{green}& & & & &\cellcolor{green}&\cellcolor{green}& & & & &\cellcolor{green}&\cellcolor{green}& & &\\ 
Task 2& & & & & &\cellcolor{cyan}&\cellcolor{cyan}& & & & &\cellcolor{cyan}&\cellcolor{cyan}& & & & &\cellcolor{cyan}&\cellcolor{cyan}& & & & &\cellcolor{cyan}&\cellcolor{cyan}&\\ 
\hline 
Task Arrivals&\cellcolor{cyan}& & & & & &\cellcolor{cyan}& & & & & &\cellcolor{cyan}& & & & & &\cellcolor{cyan}& & & & & &\cellcolor{cyan}&\\ 
\end{tabular}
}
\caption{RM Simulation results}\vspace{-1.5em}\end{table}\begin{table}\adjustbox{max height=\dimexpr\textheight-5.5cm\relax,max width=\textwidth}{\begin{tabular}{l|c|c|c|c|c|c|c|c|c|c|c|c|c|c|c|c|c|c|c|c|c|c|c|c|c|c}
 &0&1&2&3&4&5&6&7&8&9&10&11&12&13&14&15&16&17&18&19&20&21&22&23&24\\ 
 \hline 
Task 0& &\cellcolor{blue}&\cellcolor{blue}& & & & &\cellcolor{blue}&\cellcolor{blue}& & & & &\cellcolor{blue}&\cellcolor{blue}& & & & &\cellcolor{blue}&\cellcolor{blue}& & & & &\\ 
Task 1& & & &\cellcolor{green}&\cellcolor{green}& & & & &\cellcolor{green}&\cellcolor{green}& & & & &\cellcolor{green}&\cellcolor{green}& & & & &\cellcolor{green}&\cellcolor{green}& & &\\ 
Task 2& & & & & &\cellcolor{cyan}&\cellcolor{cyan}& & & & &\cellcolor{cyan}&\cellcolor{cyan}& & & & &\cellcolor{cyan}&\cellcolor{cyan}& & & & &\cellcolor{cyan}&\cellcolor{cyan}&\\ 
\hline 
Task Arrivals&\cellcolor{cyan}& & & & & &\cellcolor{cyan}& & & & & &\cellcolor{cyan}& & & & & &\cellcolor{cyan}& & & & & &\cellcolor{cyan}&\\ 
\end{tabular}
}
\caption{EDF Simulation results}\vspace{-1.5em}\end{table}\begin{table}\adjustbox{max height=\dimexpr\textheight-5.5cm\relax,max width=\textwidth}{\begin{tabular}{l|c|c|c|c|c|c|c|c|c|c|c|c|c|c|c|c|c|c|c|c|c|c|c|c|c|c}
 &0&1&2&3&4&5&6&7&8&9&10&11&12&13&14&15&16&17&18&19&20&21&22&23&24\\ 
 \hline 
Task 0& &\cellcolor{blue}&\cellcolor{blue}& & & & &\cellcolor{blue}&\cellcolor{blue}& & & & &\cellcolor{blue}&\cellcolor{blue}& & & & &\cellcolor{blue}&\cellcolor{blue}& & & & &\\ 
Task 1& & & &\cellcolor{green}&\cellcolor{green}& & & & &\cellcolor{green}&\cellcolor{green}& & & & &\cellcolor{green}&\cellcolor{green}& & & & &\cellcolor{green}&\cellcolor{green}& & &\\ 
Task 2& & & & & &\cellcolor{cyan}&\cellcolor{cyan}& & & & &\cellcolor{cyan}&\cellcolor{cyan}& & & & &\cellcolor{cyan}&\cellcolor{cyan}& & & & &\cellcolor{cyan}&\cellcolor{cyan}&\\ 
\hline 
Task Arrivals&\cellcolor{cyan}& & & & & &\cellcolor{cyan}& & & & & &\cellcolor{cyan}& & & & & &\cellcolor{cyan}& & & & & &\cellcolor{cyan}&\\ 
\end{tabular}
}
\caption{LLF Simulation results}\vspace{-1.5em}\end{table}\end{frame}\end{document}
