\documentclass{beamer}\mode<presentation> {\usetheme{Madrid}}\usepackage{booktabs,adjustbox}\usepackage{xcolor,colortbl}\title[RM, EDF, LLF]{Proyecto 2: Calendarizaci\'on en Tiempo Real}\author{Vargas A, Camacho A, Morales V}\institute[TEC]{Tecnol\'ogico de Costa Rica \medskip\textit{avargas@gmail.com, acamacho@gmail.com, verny.morales@gmail.com}\textit{3er Cuatrimestre}}\date{\today}\begin{document}\begin{frame}\titlepage\end{frame}\begin{frame}\frametitle{Rate Monotonic}\begin{block}{Tipo}Algoritmo de Scheduling Din\'amico, utilizado para la resoluci\'on de problemas ca\'oticos, como por ejemplo el problema de los carros aut\'onomos.\end{block}\begin{block}{Manejo de prioridades}Algoritmo de prioridades est\'aticas, esto quiere decir que ninguna tarea puede cambiar su prioridad. Donde la prioridad de una tarea siempre es igual a su per\'iodo. Per\'iodo mas corto, mayor la prioridad.\end{block}\begin{block}{Supuestos}Todas las tareas cr\'iticas son peri\'odicas, e independientes. El tiempo de computaci\'on se conoce a priori, y el cambio de contexto es igual a cero, o ya esta considerado en el tiempo de computaci\'on.\end{block}\end{frame}\begin{frame}\frametitle{Teoremas de Scheduling}\begin{theorem}[Par\'ametros a tomar en cuenta]$\mu = \Sigma \frac{C_{i}}{P_{i}}$ Utilizaci\'on del CPU $ Un = n 2^\frac{1}{n} - 1 $ donde n es la cantidad de tareas\end{theorem}\begin{theorem}[Condiciones de suficiencia]$\mu \leq Un $ Tareas calendarizables $\mu \geq Un$ Debido a que es una condici\'on de suficiencia podr\'ia ser calendarizable $\mu \succ 1$ Tareas no calendarizables\end{theorem}\end{frame}
\begin{frame}\frametitle{RM Scheduling Results}\begin{table}\adjustbox{max height=\dimexpr\textheight-5.5cm\relax,max width=\textwidth}{\begin{tabular}{l|c|c|c|c|c|c|c|c|c|c|c|c|c|c|c|c|c|c|c|c|c|c|c|c|c|c}
 &0&1&2&3&4&5&6&7&8&9&10&11&12&13&14&15&16&17&18&19&20&21&22&23&24\\ 
 \hline 
Task 0& &\cellcolor{blue}&\cellcolor{blue}& & &\cellcolor{blue}&\cellcolor{blue}& & &\cellcolor{blue}&\cellcolor{blue}& & &\cellcolor{blue}&\cellcolor{blue}& & &\cellcolor{blue}&\cellcolor{blue}& & &\cellcolor{blue}&\cellcolor{blue}& &\\ 
Task 1& & & &\cellcolor{green}&\cellcolor{green}& & &\cellcolor{green}&\cellcolor{green}& & &\cellcolor{green}&\cellcolor{green}& & &\cellcolor{green}&\cellcolor{green}& & &\cellcolor{green}&\cellcolor{green}& & &\cellcolor{green}&\\ 
\hline 
Task Arrivals&\cellcolor{red}& & & & & & & & & & & & & & & & & & & & & & & & &\\ 
\end{tabular}
}
\caption{RM Simulation results}\vspace{-1.5em}\end{table}\end{frame}\end{document}
