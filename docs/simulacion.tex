\documentclass{beamer}\mode<presentation> {\usetheme{Madrid}}\usepackage{booktabs,adjustbox}\usepackage{xcolor,colortbl}\title[RM, EDF, LLF]{Proyecto 2: Calendarizaci\'on en Tiempo Real} \author{Vargas A, Camacho A, Morales V} \institute[TEC] {Tecnol\'ogico de Costa Rica \medskip\textit{avargas@gmail.com, acamacho@gmail.com, verny.morales@gmail.com}\textit{3er Cuatrimestre}}\date{\today} \begin{document}\begin{frame}\titlepage\end{frame}\begin{frame}\frametitle{Early Deadline First}\begin{block}{Tipo}Algoritmo de Scheduling Din\'amico, en donde las tareas son peri\'odicas. Se considera un algoritmo \'optimo para algoritmos de prioridades din\'amicas. Es un algoritmo de tipo expropiativo.\end{block}\begin{block}{Manejo de prioridades}El nombre del algoritmo indica la pol\'itica de prioridad. La prioridad es inversamente proporcional al tiempo que falta para el deadline. El deadline de cada tarea es igual al per\'iodo de la misma. \end{block}\begin{block}{Supuestos}Todas las tareas cr\'iticas son peri\'odicas, e independientes. El tiempo de computaci\'on se conoce a priori, y el cambio de contexto es igual a cero, o ya esta considerado en el tiempo de computaci\'on.\end{block}\end{frame}\begin{frame}\frametitle{Teoremas de Scheduling}\begin{theorem}[Par\'ametros a tomar en cuenta]$\mu = \Sigma \frac{C_{i}}{P_{i}}$ Utilizaci\'on del CPU\end{theorem}\begin{theorem}[Condiciones de suficiencia]$\mu \leq 1 $ Tareas calendarizables, ya que es una condici\'on necesario y de suficiencia\end{theorem}\end{frame}\begin{frame}\frametitle{Least Laxity First}\begin{block}{Tipo y manejo de prioridades}Es un algoritmo donde las prioridades se manejan de forma din\'amica. Por cada unidad de tiempo se deben de evaluar las prioridades para as\'i conocer el Laxity de cada tarea, y tomar una decisi\'on sobre el scheduling de las mismas.\end{block}\end{frame}\begin{frame}\frametitle{Teoremas de Scheduling}\begin{theorem}[Par\'ametros a tomar en cuenta]$ Laxity = D_{i} - T_{i} + C_{i} $ Donde D es el deadline pr\'oximo de la tarea, T es el tiempo actual de ejecuci\'on y C es el tiempo computaci\'on al faltante de la tarea \end{theorem}\begin{theorem}[Criterio de Scheduling]Se toma el $ L_{i} $ menor entre todas las tareas en la cola de Ready y se ejecuta la tarea para cada tiempo de ejecuci\'on\end{theorem}\end{frame}
\begin{frame}\frametitle{EDF Prueba de Calendarizabilidad}\begin{block}{Prueba de Schedulability} $\mu=1,250000 \leq 1 $ \end{block}Las pruebas NO calendarizables\end{frame}\begin{frame}\frametitle{EDF Resultados de Simulador}\begin{table}\adjustbox{max height=\dimexpr\textheight-5.5cm\relax,max width=\textwidth}{\begin{tabular}{l|c|c|c|c|c|c|c|c|c|c|c|c|c|c|c|c|c|c|c|c|c|c|c|c|c|c}
 &0&1&2&3&4&5&6&7&8&9&10&11&12&13&14&15&16&17&18&19&20&21&22&23&24\\ 
 \hline 
Task 0& &\cellcolor{blue}&\cellcolor{blue}&\cellcolor{blue}& &\cellcolor{red}& & & & & & & & & & & & & & & & & & &\\ 
Task 1& & & & &\cellcolor{green}&\cellcolor{red}& & & & & & & & & & & & & & & & & & &\\ 
\hline 
Task Arrivals&\cellcolor{green}& & & &\cellcolor{green}&\cellcolor{red}& & & & & & & & & & & & & & & & & & & &\\ 
\end{tabular}
}
\caption{EDF Simulation results}\vspace{-1.5em}\end{table}\end{frame}\begin{frame}\frametitle{LLF Prueba de Calendarizabilidad}\begin{block}{Prueba de Schedulability} $\mu=1,250000 \leq 1 $ \end{block}Las pruebas NO calendarizables\end{frame}\begin{frame}\frametitle{LLF Resultados de Simulador}\begin{table}\adjustbox{max height=\dimexpr\textheight-5.5cm\relax,max width=\textwidth}{\begin{tabular}{l|c|c|c|c|c|c|c|c|c|c|c|c|c|c|c|c|c|c|c|c|c|c|c|c|c|c}
 &0&1&2&3&4&5&6&7&8&9&10&11&12&13&14&15&16&17&18&19&20&21&22&23&24\\ 
 \hline 
Task 0& & & &\cellcolor{blue}&\cellcolor{blue}&\cellcolor{red}& & & & & & & & & & & & & & & & & & &\\ 
Task 1& &\cellcolor{green}&\cellcolor{green}& & &\cellcolor{red}& & & & & & & & & & & & & & & & & & &\\ 
\hline 
Task Arrivals&\cellcolor{green}& & & &\cellcolor{green}&\cellcolor{red}& & & & & & & & & & & & & & & & & & & &\\ 
\end{tabular}
}
\caption{LLF Simulation results}\vspace{-1.5em}\end{table}\end{frame}\end{document}
